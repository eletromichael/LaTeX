\documentclass[12pt,a4paper]{article} %define a classe do documento como artigo, com tamanho de fonte 12pt e formato de papel A4 (210 mm de largura por 297 mm de altura).

\usepackage[portuguese]{babel} %para usar o portuguêsno documento, ajustando automaticamente a hifenização e outros aspectos linguísticos. Exemplo: separação silábica de palavras no final da linha 

\usepackage{ragged2e} %justifica o texto sem problemas de espaçamento excessivo entre as palavras.

\usepackage{graphicx} %inclusão de imagens no documento

\usepackage[colorlinks=true,        
allcolors = blue,  
citecolor=blue]{hyperref} %ativa a criação de hyperlinks no documento (como links para URLs, citações, seções). As cores dos links são definidas como preto para todos os links e azul para citações.

\usepackage[brazilian,hyperpageref]{backref} %adiciona back-references na bibliografia, indicando em quais páginas as referências foram citadas. O parâmetro hyperpageref cria links para as páginas onde as citações aparecem.

\usepackage[alf]{abntex2cite} %carrega o pacote de citações no estilo ABNT no formato alfabético, usado para citações e referências conforme o padrão brasileiro.

\renewcommand{\backrefpagesname}{Citado na(s) página(s):~}
%texto padrão antes do número das páginas
\renewcommand{\backref}{}
%define os textos da citação
\renewcommand*{\backrefalt}[4]{
	\ifcase #1 
		Nenhuma citação no texto.
	\or
		Citado na \textit{página} #2.
	\else
		Citado #1 vezes nas \textit{páginas} #2.
	\fi}
 
\begin{document}

\section*{Lei de Ohm}

\begin{justify}

\hspace{0.5cm}A Lei de Ohm é uma das leis fundamentais da eletricidade, que descreve a relação entre tensão $V$, corrente $I$ e resistência $R$ em um circuito elétrico \cite{ufs2016}. Ela afirma que a corrente que flui por um condutor é diretamente proporcional à tensão aplicada e inversamente proporcional à resistência do condutor \cite{sadiku2015}.

Se você aumentar a tensão aplicada em um circuito enquanto a resistência permanece constante, a corrente aumentará proporcionalmente. Da mesma forma, se aumentar a resistência, a corrente diminuirá, mantendo a tensão constante. A Lei de Ohm é amplamente utilizada para projetar e analisar circuitos elétricos e eletrônicos \cite{boylestad2016}.
\end{justify}

\newpage 
\bibliography{referencias}

\end{document}
