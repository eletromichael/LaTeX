%---------------------------------------------
\documentclass[hyperref={colorlinks=true,    
allcolors = blue,citecolor=blue}]{beamer} %Define a classe de documento, usada para criar apresentações. A opção hyperref especifica que os links internos devem ter cor azul e que as citações também aparecem em azul.
%---------------------------------------------
\usepackage[portuguese]{babel} %Define o idioma e o método de separação silábica no final da linha
%---------------------------------------------
\usetheme{Madrid} %Tema da apresentação
%---------------------------------------------
\useoutertheme{miniframes} %Aplica um estilo de moldura externa chamado miniframes, que adiciona pequenas molduras nos slides para melhorar a navegação visual.
%---------------------------------------------
\usepackage{graphicx} %Inserção de figuras
%---------------------------------------------
\usepackage{tikz} %Pacote para desenhos gráficos vetoriais, permitindo a criação de diagramas, formas e gráficos 
%---------------------------------------------
\usepackage{subfig} %Permite a inclusão de figuras ou subfiguras dentro de um mesmo ambiente, útil para organizar múltiplas imagens em um único quadro.
%---------------------------------------------
\usepackage{subcaption} %Oferece mais controle sobre legendas para subfiguras dentro de uma figura principal.
%---------------------------------------------
\usepackage[table]{xcolor} %Permite a coloração de tabelas em LaTeX, oferecendo funcionalidades adicionais para manipulação de cores dentro de tabelas.
%---------------------------------------------
\usepackage{float} %Fornece controle avançado para o posicionamento de elementos flutuantes, como figuras e tabelas, permitindo usar o modificador [H] para definir posição exata no documento.
%---------------------------------------------
\usepackage{array} %Fornece ferramentas avançadas para o formato de colunas e tabelas, permitindo definir, por exemplo, o alinhamento e o espaçamento de colunas de tabelas de maneira mais flexível.
%---------------------------------------------
%Definir exatamente a cor desejada, a estrutura é {nomedacor}{RGB}{nº de Red, nº de Green, nº de Blue}
\definecolor{vermelho}{RGB}{229,34,21}
\definecolor{azul}{RGB}{18,115,173}
%---------------------------------------------
\usepackage[brazilian,hyperpageref]{backref} %adiciona back-references na bibliografia, indicando em quais páginas as referências foram citadas. O parâmetro hyperpageref cria links para as páginas onde as citações aparecem.
%---------------------------------------------
\usepackage[alf]{abntex2cite} %carrega o pacote de citações no estilo ABNT no formato alfabético, usado para citações e referências conforme o padrão brasileiro.
%---------------------------------------------



\renewcommand{\backrefpagesname}{Citado nos(s) slide(s):~}
%texto padrão antes do número dos slides
\renewcommand{\backref}{}
%define os textos da citação
\renewcommand*{\backrefalt}[4]{
	\ifcase #1 
		Nenhuma citação no texto.
	\or
		Citado no \textit{slide} #2.
	\else
		Citado #1 vezes nos \textit{slides} #2.
	\fi}

%--------------------------------------------- 
\renewcommand{\arraystretch}{1.6}%entre linhas das células 
%---------------------------------------------
\setbeamertemplate{caption}[numbered] %numera automaticamente figuras e tabelas
%---------------------------------------------
\setbeamercolor{palette primary}{bg=vermelho,fg=white} %No rodapé da direita para esquerda, primeiro campo, onde está declarada a data. Fonte branca e fundo vermelho.
%---------------------------------------------
\setbeamercolor{palette secondary}{bg=azul,fg=white} %No rodapé da direita para esquerda, no segundo campo, onde está o e-mail. Fonte azul (escura "blue" do xcolor) e fundo azul (que foi definido).
%---------------------------------------------
\setbeamercolor{palette tertiary}{bg=azul,fg=white} %Topo do sumário simplificado. Fonte branca e fundo azul.
%---------------------------------------------
\setbeamercolor{structure}{fg=vermelho} %Cor de itens de um 'itemize'
%---------------------------------------------
\setbeamercolor{section in toc}{fg=azul} %Cor dos títulos das seções
%---------------------------------------------
%Cor das subseções, fonte e fundo.
\setbeamercolor{subsection in head/foot}{bg=black,fg=white}
%---------------------------------------------
%Cabeçalho
%---------------------------------------------
\title[seuemail@gmail.com]{Título da apresentação}
\date{Cidade, XX de Mês de 202X.}
\author[Sigla da Faculdade]{Seu nome \\ \vspace{0.59cm}
Código da disciplina - Nome da disciplina Ano/Semestre\\
Prof./Profª. Dr./Profª. Nome do(a) docente/orientador(a)}
%---------------------------------------------
%Logotipo da universidade na capa da apresentação 
%---------------------------------------------
\titlegraphic{
        \begin{tikzpicture}
            \draw (-8,5)--(-5,5)--(-5,4)--(-8,4)--(-8,5);
            \draw (-7.5,4.5) node[right]{Logotipo};
        \end{tikzpicture}
}

%O ambiente tikz deve ser substituído pelo \includegraphics[]{} adicionando uma imagem que esteja na raíz do arquivo main ou numa pasta específica de imagens.
%---------------------------------------------
\begin{document} 
%---------------------------------------------
%Exibição da capa do Beamer
%---------------------------------------------
\begin{frame}
	\titlepage	
\end{frame}
%---------------------------------------------
%Sumário longo é usado como duas colunas 
%---------------------------------------------
\begin{frame}
    \begin{columns}[onlytextwidth,T]
    \begin{column}{.45\textwidth}
    \tableofcontents[sections=1-4]
    \end{column}
    \begin{column}{.45\textwidth}
    \tableofcontents[sections=5-9]
\end{column}    
\end{columns}
\end{frame}	
%-----------------------------
\begin{frame}
\section{Contextualização}
\frametitle{Contextualização}

\begin{itemize}
    \item Aspecto histórico.
    \item Descrever o cenário em que o projeto se insere.
\end{itemize}

\end{frame}
%-----------------------------
\begin{frame}
\subsection{Situação-problema}
\frametitle{Situação-problema}

%Duas colunas: esquerda texto e direita uma imagem, o código \begin{tikzpicture} até \end{tikzpicture} deve ser substituído por \includegraphics[]{}

\begin{columns}
    \begin{column}{0.47\textwidth}
        \begin{itemize}
            \item Problema.
            \item Estatísticas e dados confiáveis.
            \item O que foi realizado nos últimos 5 anos para amenizar o problema?
        \end{itemize}
    \end{column}
    \begin{column}{0.5\textwidth}
        \begin{figure}
            \centering            
        \begin{tikzpicture}
         \draw (-7,5)--(-7,2)--(-4,2)--(-4,5)--(-7,5);     
        \end{tikzpicture}
        \caption{Uma figura autoexplicativa, \cite{geogebra}.}
        \label{fig:enter-label}
        \end{figure}
    \end{column}
\end{columns}

\end{frame}
%-----------------------------
\begin{frame}
\subsection{Justificativa}
\frametitle{Justificativa}

Explique de maneira clara e concisa por que o projeto é necessário e qual é a relevância dele no contexto do seu campo de estudo.

\end{frame}
%-----------------------------
\begin{frame}
\section{Objetivos}
\subsection{Objetivo geral}
\frametitle{Objetivo geral}

Uma única frase que represente o objetivo central do projeto.

Palavras-chave: \textbf{desenvolver}, \textbf{implementar}, \textbf{projetar}, \textbf{criar}.

\end{frame}
%-----------------------------
\begin{frame}
\subsection{Objetivos específicos}
\frametitle{Objetivo específicos}

\begin{itemize}
    \item Objetivo específico 1
    \item Objetivo específico 2
    \item Objetivo específico 3
    \item Objetivo específico 4
    \item Objetivo específico 5    
\end{itemize}

Palavras-chave: \textbf{analisar}, \textbf{simular}, \textbf{verificar}, \textbf{caracterizar}, \textbf{comparar}, \textbf{avaliar}.

\end{frame}
%-----------------------------
\begin{frame}
\section{Cronograma}
\frametitle{Cronograma}

%Tabela com espaçamento adequado

\begin{table}[!htp]
\centering
\begin{tabular}{ | m{1.5cm} | m{25em}| } 
  \hline
  Mês & Atividades \\ 
  \hline
  Janeiro & \begin{itemize}
    \item Definição dos objetivos do projeto.
    \item  Requisitos técnicos e funcionais
    \item  Levantamento de soluções existentes.
    \item Escolha de ferramentas (\textit{software} CAD, simulação).\end{itemize} \\  \hline
  Fevereiro  & \begin{itemize}      
\item Diagrama de blocos e definição da arquitetura.
\item Modelagem e simulação (SPICE, ModelSim).
\item Verificação de viabilidade técnica. 
  \end{itemize}  \\  \hline  
\end{tabular}
\caption{Cronograma: parte 1, \cite{nome_do_autor}.}
\label{tabela1}
\end{table}

\end{frame}
%-----------------------------
\begin{frame}
\frametitle{Cronograma}

\begin{table}[!htp]
\centering
\begin{tabular}{ | m{1.5cm} | m{25em}| } 
  \hline
  Mês & Atividades \\ 
  \hline
  Março & \begin{itemize}
	\item Projeto detalhado do circuito.
	\item Desenvolvimento do \textit{layout} PCB ou microeletrônico.
	\item Uso de ferramentas CAD (Altium, KiCad, Virtuoso).
\end{itemize} \\  \hline
  Abril  & \begin{itemize}
	\item Revisão do layout.
	\item Envio para fabricação do protótipo de PCB ou fabricação em tecnologia específica (CMOS, FPGA).
	\item Aquisição de componentes.
\end{itemize}  \\  \hline  
\end{tabular}
\caption{Cronograma: parte 2, \cite{nome_do_autor}.}
\label{tabela2}
\end{table}

\end{frame}
%-----------------------------
\begin{frame}
\frametitle{Cronograma}

\begin{table}[!htp]
\centering
\begin{tabular}{ | m{1.5cm} | m{25em}| } 
  \hline
  Mês & Atividades \\ 
  \hline
  Maio & 
\begin{itemize}
	\item Montagem do protótipo e testes de desempenho.	
	\item Validação dos parâmetros elétricos.
	\item Debug de eventuais problemas no hardware.
\end{itemize} \\  \hline
  Junho & \begin{itemize}
	\item Elaboração da documentação técnica.
	\item Relatório de resultados e análise comparativa.
	\item Preparação da apresentação final.
	\item Discussão de melhorias e trabalhos futuros.
\end{itemize}   \\  \hline  
\end{tabular}
\caption{Cronograma: parte 3, \cite{nome_do_autor}.}
\label{tabela2}
\end{table}

\end{frame}
%-----------------------------
\begin{frame}
\subsection{Orçamento}
\frametitle{Orçamento}

\begin{table}[!htp]
\centering
\begin{tabular}{ | m{2.5cm} | m{2.5cm}| m{2.5cm}| m{2.5cm}| } 
  \hline
  Componentes & Empresa A & Empresa B & Empresa C \\ 
  \hline
  Componente 1 & R\$ 1,00 & R\$ 2,00 & R\$ 3,00 \\  \hline
  Componente 2 & R\$ 2,00  & R\$ 4,00 & R\$ 6,00  \\  \hline
  Componente 3 & R\$ 3,00 & R\$ 8,00 & R\$ 9,00 \\  \hline
  Componente 4 & R\$ 4,00 & R\$  10,00 & R\$ 12,00  \\  \hline
  Componente 5 & R\$ 5,00  & R\$ 12,00 & R\$ 15,00 \\  \hline 
  \rowcolor{yellow}Total & R\$ 15,00  & R\$ 36,00 & R\$ 45,00 \\  \hline  
\end{tabular}
\caption{Orçamento realizado para o projeto, \cite{radional_2024, eletronica_system_2024, loja_severo_roth_2024}.}
\label{tabela2}
\end{table}


\end{frame}
%-----------------------------
\begin{frame}
\section{Metodologia}
\frametitle{Metodologia}

Apresente detalhadamente como o projeto será realizado. Descrevendo as etapas do trabalho, as ferramentas, as técnicas e os procedimentos que serão usados para alcançar os objetivos do projeto. O objetivo desse \textit{slide} é demonstrar a clareza do caminho escolhido para solucionar o problema ou desenvolver a solução.

\end{frame}
%-----------------------------
\begin{frame}
\subsection{Desenvolvimento mecânico}
\frametitle{Desenvolvimento mecânico}

%Duas colunas: esquerda texto e direita uma imagem, o código \begin{tikzpicture} até \end{tikzpicture} deve ser substituído por \includegraphics[]{}

\begin{columns}
    \begin{column}{0.47\textwidth}
        \begin{itemize}
            \item Tamanhos e medidas.
            \item \textit{Layout}.
            \item Principais peças.
        \end{itemize}
    \end{column}
    \begin{column}{0.5\textwidth}
        \begin{figure}
            \centering            
        \begin{tikzpicture}
         \draw (-7,5)--(-7,2)--(-4,2)--(-4,5)--(-7,5);     
        \end{tikzpicture}
        \caption{Protótipo do projeto, \cite{tinkercad}.}
        \label{fig:enter-label}
        \end{figure}
    \end{column}
\end{columns}


\end{frame}
%-----------------------------
\begin{frame}
\subsection{Desenvolvimento de hardware}
\frametitle{Desenvolvimento de \textit{hardware}}

%Duas colunas: esquerda texto e direita uma imagem, o código \begin{tikzpicture} até \end{tikzpicture} deve ser substituído por \includegraphics[]{}

\begin{columns}
    \begin{column}{0.47\textwidth}
        \begin{itemize}
            \item Diagrama de blocos.
            \item Esquemático elétrico.
            \item Principais componentes.
        \end{itemize}
    \end{column}
    \begin{column}{0.5\textwidth}
        \begin{figure}
            \centering            
        \begin{tikzpicture}
         \draw (-7,5)--(-7,2)--(-4,2)--(-4,5)--(-7,5);     
        \end{tikzpicture}
        \caption{Diagrama elétrico completo, \cite{tinati}.}
        \label{fig:enter-label}
        \end{figure}
    \end{column}
\end{columns}

\end{frame}
%-----------------------------
\begin{frame}
\subsection{Desenvolvimento de software}
\frametitle{Desenvolvimento de \textit{software}}

%Duas colunas: esquerda texto e direita uma imagem, o código \begin{tikzpicture} até \end{tikzpicture} deve ser substituído por \includegraphics[]{}

\begin{columns}
    \begin{column}{0.47\textwidth}
        \begin{itemize}
            \item Plataforma de desenvolvimento.
            \item Linguagem de programação simplificada.      
        \end{itemize}
    \end{column}
    \begin{column}{0.5\textwidth}
        \begin{figure}
            \centering            
        \begin{tikzpicture}
         \draw (-7,5)--(-7,2)--(-4,2)--(-4,5)--(-7,5);     
        \end{tikzpicture}
        \caption{Arduino IDE, \cite{arduinoIDE}.}
        \label{fig:enter-label}
        \end{figure}
    \end{column}
\end{columns}

\end{frame}
%-----------------------------
\begin{frame}[fragile]
\subsection{Desenvolvimento de firmware}
\frametitle{Desenvolvimento de \textit{firmware}}

%\begin{verbatim} até \end{verbatim} carrega um código específico de programação

\begin{verbatim}
void setup() {  
  pinMode(LED_BUILTIN, OUTPUT);
}

void loop() {
  digitalWrite(LED_BUILTIN, HIGH);  
  delay(1000);                      
  digitalWrite(LED_BUILTIN, LOW);   
  delay(1000);                      
}
\end{verbatim}


\end{frame}
%-----------------------------
\begin{frame}
\section{Modelagem matemática}
\frametitle{Modelagem matemática}

\begin{itemize}
    \item Explique brevemente o conceito de modelagem matemática no contexto da engenharia elétrica/eletrônica.
    \item A modelagem matemática consiste em representar sistemas físicos e circuitos elétricos por meio de equações e expressões matemáticas para analisar seu comportamento e desempenho.
\end{itemize}

\end{frame}
%-----------------------------
\begin{frame}
\subsection{Parâmetros e hipóteses}
\frametitle{Parâmetros e hipóteses}

\begin{itemize}
    \item Os parâmetros são valores ou constantes que caracterizam as propriedades físicas e operacionais do sistema ou do circuito. Eles são essenciais para a construção das equações matemáticas e influenciam diretamente o comportamento do modelo.
\end{itemize}

\end{frame}
%-----------------------------
\begin{frame}
\subsection{Resolução do modelo}
\frametitle{Resolução do modelo}

\begin{itemize}
    \item Mostre como as equações matemáticas derivadas da modelagem são resolvidas para obter respostas práticas sobre o comportamento do sistema. A resolução do modelo envolve aplicar técnicas matemáticas e computacionais para analisar as equações e encontrar as variáveis de interesse (tensão, corrente, potência) em diferentes condições de operação.
    \item Explique se a resolução é analítica, numérica ou através de simulação computacional.
\end{itemize}

\end{frame}
%-----------------------------
\begin{frame}
\subsection{Validação}
\frametitle{Validação}

\begin{itemize}
    \item Inclua gráficos de resultados obtidos a partir da solução analítica, numérica ou simulação.
    \item Resposta no tempo (curvas de tensão e corrente).
    \item Resposta em frequência.
    \item Curvas de potência ou eficiência energética.
\end{itemize}

\end{frame}
%-----------------------------
\begin{frame}
\subsection{Análise de resultados}
\frametitle{Análise de resultados}

Mostre a diferença entre uma solução analítica (resposta exata) e a solução numérica aproximada para um circuito com componentes não-ideais.

\end{frame}
%-----------------------------
\begin{frame}
\section{Conclusão}
\frametitle{Conclusão}

Destaque os resultados mais relevantes obtidos no projeto. Mostre como o trabalho contribui para a área ou resolve um problema específico. Destaque a relevância prática e acadêmica.

\end{frame}
%-----------------------------
\begin{frame}
\subsection{Considerações finais}
\frametitle{Considerações finais}

Retomada dos objetivos específicos, identificar quais foram atingidos e quais não foram atingidos. Elaborar uma discussão e encaminhar trabalhos futuros.

\end{frame}
%-----------------------------
%Se excluir a opção do slide de permissão de quebras de quadro vai dar erro na exibição das referências bibliográficas.
%-----------------------------
\begin{frame}[allowframebreaks]
\section{Referências bibliográficas}
\frametitle{Referências bibliográficas}  
      
    \bibliography{referenciais_b}
    
\end{frame}
%-----------------------------
\end{document}
